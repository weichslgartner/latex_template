%\makeatletter
%\def\pgfplots@getautoplotspec into#1{%
    %\begingroup
    %\let#1=\pgfutil@empty
    %\pgfkeysgetvalue{/pgfplots/cycle multi list/@dim}\pgfplots@cycle@dim
    %%
    %\let\pgfplots@listindex=\pgfplots@numplots
    %%%% Start new code
    %\pgfkeysgetvalue{/pgfplots/cycle list set}\pgfplots@listindex@set
    %\ifx\pgfplots@listindex@set\pgfutil@empty
    %\else 
        %\c@pgf@counta=\pgfplots@listindex
        %\c@pgf@countb=\pgfplots@listindex@set
        %\advance\c@pgf@countb by -\c@pgf@counta
        %\globaldefs=1\relax
        %\edef\setshift{%
            %\noexpand\pgfkeys{
                %/pgfplots/cycle list shift=\the\c@pgf@countb,
                %/pgfplots/cycle list set=
            %}
        %}%
        %\setshift%
    %\fi
    %%%% End new code    
    %\pgfkeysgetvalue{/pgfplots/cycle list shift}\pgfplots@listindex@shift
    %\ifx\pgfplots@listindex@shift\pgfutil@empty
    %\else
        %\c@pgf@counta=\pgfplots@listindex\relax
        %\advance\c@pgf@counta by\pgfplots@listindex@shift\relax
        %\ifnum\c@pgf@counta<0
            %\c@pgf@counta=-\c@pgf@counta
        %\fi
        %\edef\pgfplots@listindex{\the\c@pgf@counta}%
    %\fi
    %\ifnum\pgfplots@cycle@dim>0
        %% use the 'cycle multi list' feature.
        %%
        %% it employs a scalar -> multiindex map like
        %% void fromScalar( size_t d, size_t scalar, size_t* Iout, const size_t* N )
        %% {
        %%   size_t ret=scalar;
        %%   for( int i = d-1; i>=0; --i ) {
        %%       Iout[i] = ret % N[i];
        %%       ret /= N[i];
        %%   }
        %% }
        %% to get the different indices into the cycle lists.
        %%-------------------------------------------------- 
        %\c@pgf@counta=\pgfplots@cycle@dim\relax
        %\c@pgf@countb=\pgfplots@listindex\relax
        %\advance\c@pgf@counta by-1
        %\pgfplotsloop{%
            %\ifnum\c@pgf@counta<0
                %\pgfplotsloopcontinuefalse
            %\else
                %\pgfplotsloopcontinuetrue
            %\fi
        %}{%
            %\pgfkeysgetvalue{/pgfplots/cycle multi list/@N\the\c@pgf@counta}\pgfplots@cycle@N
            %% compute list index:
            %\pgfplotsmathmodint{\c@pgf@countb}{\pgfplots@cycle@N}%
            %\divide\c@pgf@countb by \pgfplots@cycle@N\relax
            %%
            %\expandafter\pgfplots@getautoplotspec@
                %\csname pgfp@cyclist@/pgfplots/cycle multi list/@list\the\c@pgf@counta @\endcsname
                %{\pgfplots@cycle@N}%
                %{\pgfmathresult}%
            %\t@pgfplots@toka=\expandafter{#1,}%
            %\t@pgfplots@tokb=\expandafter{\pgfplotsretval}%
            %\edef#1{\the\t@pgfplots@toka\the\t@pgfplots@tokb}%
            %\advance\c@pgf@counta by-1
        %}%
    %\else
        %% normal cycle list:
        %\pgfplotslistsize\autoplotspeclist\to\c@pgf@countd
        %\pgfplots@getautoplotspec@{\autoplotspeclist}{\c@pgf@countd}{\pgfplots@listindex}%
        %\let#1=\pgfplotsretval
    %\fi
    %\pgfmath@smuggleone#1%
    %\endgroup
%}
%


\makeatletter
\newcommand\resetstackedplots{
	\makeatletter
	\pgfplots@stacked@isfirstplottrue
	\makeatother
	\addplot [forget plot,draw=none] coordinates{(1,0) (2,0) (3,0) (4,0)(5,0)(6,0)(7,0)(8,0)};
}
\makeatother
